\documentclass[10pt,a4paper]{article}
\usepackage[utf8]{inputenc}
\usepackage[french]{babel}
\usepackage[T1]{fontenc}
\usepackage{geometry}
\geometry{margin=2cm}
\usepackage{setspace}
\setstretch{1.15}
\setlength{\parindent}{0pt}

\begin{document}

\begin{center}
\Large\textbf{Règlement intérieur de SOLECTRO}
\end{center}

\vspace{1em}

Ce règlement intérieur a pour objectif de préciser les statuts de l'association Solectro, dont l'objet est « Gestion et Organisation d'événements audiovisuels ».

Il sera remis à l'ensemble des membres ainsi qu'à chaque nouvel adhérent.

\vspace{1em}

\section*{Titre I : Membres}

\subsection*{Article 1 - Cotisation}
Les membres ne paient pas de cotisation (sauf s'ils en décident autrement de leur propre volonté).

Si un membre décide de son plein gré de payer une cotisation alors le montant de celle-ci est libre.

Toute cotisation versée à l'association est définitivement acquise. Aucun remboursement de cotisation ne peut être exigé en cas de démission, d'exclusion ou de décès d'un membre en cours d'année.

\subsection*{Article 2 - Admission de nouveaux adhérents}
L'association Solectro peut à tout moment accueillir de nouveaux adhérents. La qualité d'adhérent s'acquiert sous deux conditions :
\begin{itemize}
\item Avoir un accord écrit des membres du bureau
\item Avoir signé et accepté le présent règlement
\end{itemize}

\subsection*{Article 3 - Exclusion}
Selon la procédure définie à l'article 8 des statuts de l'association Solectro, seuls les cas d'inactivité pendant plus de 5 ans, de destruction et/ou de détérioration volontaire de matériel, incivilités envers un ou plusieurs adhérents de l'association et/ou du public et/ou des collaborateurs de l'association, la mise en danger d'autrui et le non-respect du règlement intérieur peuvent déclencher une procédure d'exclusion.

Celle-ci doit être prononcée par le bureau à une majorité, seulement après avoir entendu les explications du membre contre lequel une procédure d'exclusion est engagée.

Si l'exclusion est prononcée, un recours peut être engagé par le membre concerné.

\subsection*{Article 4 – Démission, Décès, Disparition}
Le membre démissionnaire devra adresser sous lettre ou par courrier électronique sa décision au bureau.

Le membre démissionnaire ne peut prétendre à une restitution de cotisation.

En cas de décès ou de disparition, la qualité de membre s'efface avec la personne.

\section*{Titre II : Fonctionnement de l'association}

\subsection*{Article 5 - Le bureau}
Conformément à l'article 13 des statuts de l'association Solectro, le bureau a pour objet de diriger l'association.

Il est composé de M. Bessiere Lucas ; Delaître Romain ; Delgado Mathis

\subsection*{Article 6 - Assemblée Générale Ordinaire}
Conformément à l'article 10. des statuts de l'association Solectro, l'Assemblée Générale Ordinaire se réunit 1 fois par an sur convocation du bureau.

Tous les membres sont autorisés à participer.

Ils sont convoqués suivant la procédure suivante :

Envoi d'une convocation par mail.

Le vote des résolutions s'effectue par vote à main levée.

Déroulement d'une assemblée générale :
\begin{itemize}
\item Énoncé de l'ordre du jour
\item Discussion autour des sujets à l'ordre du jour
\item Vote à main levée des propositions
\item Discussion des sujets hors de l'ordre du jour
\end{itemize}

\subsection*{Article 7 - Assemblée Générale Extraordinaire}
Conformément à l'article 11 des statuts de l'association Solectro, une Assemblée Générale Extraordinaire peut être convoquée sur la demande du président ou sur la demande de plus de la moitié des membres inscrits afin de discuter des modifications des statuts ou la dissolution ou pour des actes portant sur des violations de règlement intérieur.

L'ensemble des membres de l'association seront convoqués selon la procédure suivante :

Envoi d'une convocation par mail.

Le vote se déroule selon les modalités suivantes : vote à main levée.

Les votes par procuration ou par correspondance sont autorisés

\section*{Titre III : Dispositions diverses}

\subsection*{Article 8 - Modification du règlement intérieur}
Le règlement intérieur de l'association Solectro est établi par le bureau, conformément à l'article 15 des statuts.

Il peut être modifié par le bureau, sur proposition de plus de la moitié des membres.

Le nouveau règlement intérieur sera adressé à chacun des membres de l'association par mail sous un délai de 15 jours suivant la date de la modification.

\subsection*{Article 9 – Remboursement des dégâts volontaires}
Suite à une détérioration volontaire du matériel par l'un des membres de l'association Solectro, un remboursement intégral peut être exigé si l'assurance ne peut prendre en charge le dégât.

\subsection*{Article 10 – Vol}
Il est formellement interdit de voler le matériel ou les denrées appartenant à l'association, une réparation peut être exigée et/ou des poursuites pourront être engagées suivant la gravité du vol.

\vspace{2cm}

\begin{minipage}{0.49\linewidth}
\begin{flushleft}
Nom, Prénom, Date :
\end{flushleft}
\end{minipage}
\begin{minipage}{0.49\linewidth}
\begin{flushright}
Signature précédé de la mention "Lu et approuvé": 
\end{flushright}
\end{minipage}






\end{document}