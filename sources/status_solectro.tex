\documentclass[10pt,a4paper]{article}
\usepackage[utf8]{inputenc}
\usepackage[french]{babel}
\usepackage[T1]{fontenc}
\usepackage{geometry}
\geometry{margin=2cm}
\usepackage{setspace}
\setstretch{1.15}
\setlength{\parindent}{0pt}

\begin{document}

\begin{center}
\Large\textbf{Déclaration de statuts}\\[0.5em]
\textbf{\normalsize Proposé aux associations déclarées par application de la\\
loi du 1\textsuperscript{er} juillet 1901 et du décret du 16 août 1901.}
\end{center}

\vspace{1em}

\section*{ARTICLE PREMIER - NOM}
Il est fondé entre les adhérents aux présents statuts une association régie par la loi du 1\textsuperscript{er} juillet 1901 et le décret du 16 août 1901, ayant pour titre : \textbf{Solectro}

\section*{ARTICLE 2 - BUT OBJET}
Gestion et organisation d'événements, et conception de solutions audiovisuelles et informatiques. Lors de ces dits événements, une buvette payante pourra être ouverte. Selon la nature des prestations fournies, l’association pourra percevoir une contribution destinée à assurer ses frais de fonctionnement et ses investissements.

\section*{ARTICLE 3 - SIÈGE SOCIAL}
Le siège social est fixé à la Maison des Associations de Pecqueuse, 12 Grand Rue 91470 Pecqueuse.

\section*{Article 4 - DURÉE}
La durée de l'association est illimitée.

\section*{ARTICLE 5 - COMPOSITION}
L'association se compose de :
\begin{enumerate}
\item[a)] Membres du Bureau
\item[b)] Membres actifs ou adhérents
\end{enumerate}

\section*{ARTICLE 6 - ADMISSION}
L'admission est soumise à conditions. Lesdites conditions sont détaillées dans le Règlement Intérieur.

\section*{ARTICLE 7 - MEMBRES – COTISATIONS}
La cotisation n'est pas obligatoire et son montant est libre, tel que statué dans le Règlement Intérieur.

\section*{ARTICLE 8. - RADIATIONS}
La qualité de membre se perd par :
\begin{enumerate}
\item[a)] La démission;
\item[b)] Le décès;
\item[c)] La radiation prononcée par le bureau pour motif grave, l'intéressé ayant été invité à fournir des explications devant le bureau et/ou par écrit.
\item[d)] Pour inactivité comme précisé dans le règlement intérieur.
\end{enumerate}

\section*{ARTICLE 9. - RESSOURCES}
Les ressources de l'association comprennent :
\begin{enumerate}
\item[a)] Le montant des droits d'entrée et des cotisations;
\item[b)] Les subventions de l'État, des régions, des départements et des communes.
\item[c)] Les donations personnelles sans aucune contrepartie.
\item[d)] La revente des biens de l'association.
\item[e)] Les revenus générés par les prestations fournies.
\item[f)] Toutes les ressources autorisées par les lois et règlements en vigueur.
\end{enumerate}

\section*{ARTICLE 10 - ASSEMBLÉE GÉNÉRALE ORDINAIRE}
Tous les membres de l'association, à quelques titres qu'ils soient, peuvent participer à l'assemblée générale ordinaire. Elle se réunit exactement une fois par an. Le quorum minimum pour que l'assemblée générale ordinaire soit considérée comme valide est de 10\%.

Quinze jours au moins avant la date fixée par un vote à la majorité du bureau, tous les membres de l'association sont convoqués par les soins de ce dernier. Un premier ordre du jour figure sur les convocations.

Cet ordre du jour pourra être complétée par des questions des membres de l'association dans un délai de 48h avant l'assemblée générale ordinaire et soumise à un vote du bureau.

Le président, assisté des membres du bureau, préside l'assemblée et expose la situation morale ou l'activité de l'association.

Le trésorier rend compte de sa gestion et soumet les comptes annuels (bilan, compte de résultat et annexe) à l'approbation de l'assemblée.

Ne peuvent être abordés que les points inscrits à l'ordre du jour communiqué 24h à l'avance.

Les décisions sont prises à la majorité des voix des membres présents ou représentés. Il est procédé, après épuisement de l'ordre du jour, au renouvellement des membres sortants du bureau.

Toutes les délibérations sont prises à main levée sauf si un des membres de l'association présent demande un scrutin a bulletin secret.

Les décisions des assemblées générales s'imposent à tous les membres, y compris absents ou représentés.

\section*{ARTICLE 11 - ASSEMBLÉE GÉNÉRALE EXTRAORDINAIRE}
Si besoin est, ou sur la demande de la moitié plus un des adhérents ou des membres du bureau, le président doit convoquer une assemblée générale extraordinaire dans le délai d'un mois suivant le vote, suivant les modalités prévues aux présents statuts et uniquement pour modification des statuts ou la dissolution. Le quorum minimum pour que l'assemblée générale extraordinaire soit considérée comme valide est de 20\%.

Les modalités de convocation sont les mêmes que pour l'assemblée générale ordinaire.

Les délibérations sont prises à la majorité des membres présents.

\section*{ARTICLE 12 - CONSEIL D'ADMINISTRATION}
L'association est dirigée exclusivement par le bureau, par conséquent, celle-ci ne dispose pas de Conseil d'Administration.

\section*{ARTICLE 13 – LE BUREAU}
L'assemblée générale élit parmi les membres de l'association un bureau composé de :
\begin{enumerate}
\item[a)] Un président ;
\item[b)] Des vices-présidents ;
\item[c)] Un trésorier, et, si besoin est, un secrétaire.
\end{enumerate}

\section*{ARTICLE 14 – INDEMNITÉS}
Toutes les fonctions, y compris celles des membres du bureau, sont gratuites et bénévoles. Seuls les frais occasionnés par l'accomplissement de leur mandat sont remboursés sur justificatifs. Le rapport financier présenté à l'assemblée générale ordinaire présente, par bénéficiaire, les remboursements de frais de mission, de déplacement ou de représentation.

\section*{ARTICLE - 15 - RÈGLEMENT INTÉRIEUR}
Un règlement intérieur est établi par le bureau et voté à la majorité par le bureau.

Ce règlement est destiné à fixer les divers points non prévus par les présents statuts, notamment ceux qui ont trait à l'administration interne de l'association.

\section*{ARTICLE - 16 - DISSOLUTION}
En cas de dissolution prononcée selon les modalités prévues à l'article 11, un ou plusieurs liquidateurs sont nommés, et l'actif net, s'il y a lieu, est dévolu à un organisme ayant un but non lucratif conformément aux décisions de l'assemblée générale extraordinaire qui statue sur la dissolution. L'actif net ne peut être dévolu à un membre de l'association, même partiellement, sauf reprise d'un apport.

\section*{Article – 17 - LIBÉRALITÉS :}
Le rapport et les comptes annuels, tels que définis à l'article 10 (y compris ceux des comités locaux) sont adressés chaque année au Préfet du département.

L'association s'engage à présenter ses registres et pièces de comptabilité sur toute réquisition des autorités administratives en ce qui concerne l'emploi des libéralités qu'elle serait autorisée à recevoir, à laisser visiter ses établissements par les représentants de ces autorités compétentes et à leur rendre compte du fonctionnement desdits établissements.

\vspace{2em}

\begin{flushright}
Fait à Pecqueuse, le 14/12/2025

\vspace{1em}

BESSIERE Lucas, Président\\
\end{flushright}

\end{document}
